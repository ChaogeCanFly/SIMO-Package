\let\clearforchapter\par % cheating, but saves some space
\chapter{Fundamentals of the IsoGeometric Analysis} 

\section{Univariate B-splines}

\subsection{Knot vector and B-spline functions, refinement, spline derivatives}

\subsection{B-spline curves}

\section{Multivariate tensor product B-splines}
A B-spline patch (curve, surface, volume) is essentially defined by these corresponding formulas
\begin{equation}
    \mathbf{C}(\xi) = \sum_{i = 1}^{n} N_{i,p}(\xi) \mathbf{P}_i.
\end{equation}

\begin{equation}
    \mathbf{S}(\xi,\eta)=\sum_{i=1}^n \sum_{j=1}^m N_{i,p}(\xi) M_{j,q}(\eta) \mathbf{P}_{i,j},
\end{equation}

\begin{equation}
    \mathbf{S}(\xi,\eta, \zeta)=\sum_{i=1}^n \sum_{j=1}^m \sum_{k=1}^l N_{i,p}(\xi) M_{j,q}(\eta) L_{k,l}(\zeta) \mathbf{P}_{i,j,k},
\end{equation}
where $N_{i,p}(\xi), M_{j,q}(\eta), L_{k,l}(\zeta)$ are univariate basis functions, $\mathbf{P}$ is the 4-vector of control point coordinates.
\subsection{Knot vectors, B-spline functions}

\subsection{NURBS geometries, single patch and multi--patch domains}

\section{IsoGeometric Analysis}

\subsection{Basic Idea and Fundamentals}

\subsection{Isogeometric Analysis in Detail}

\subsubsection{B-splines and NURBS as Basis Functions}

\subsection{Mesh Structure in Isogeometric Analysis}

\subsection{Geometry Mapping}

\subsection{Refinement strategies}